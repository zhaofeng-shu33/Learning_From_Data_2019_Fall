\documentclass[a4paper, 12pt, answers]{exam}
\usepackage[T1]{fontenc}
\usepackage{amsmath}
\usepackage{amssymb}
\usepackage{enumerate}
\usepackage{bm}
\usepackage{advdate}
\usepackage{datetime}
\usepackage[mathcal]{eucal}
\usepackage{dsfont}
\usepackage[numbered,framed]{matlab-prettifier}
\usepackage{tkz-euclide}
\usepackage[colorlinks,urlcolor=blue]{hyperref}
\usepackage{graphicx}

\usetkzobj{all}
\usepackage{url}
\newdate{issuedate}{25}{10}{2019}
\newdate{duedate}{10}{11}{2019}

% \newcommand{\duedate}[1][14]{%
% \begingroup
% \AdvanceDate[#1]%
% \today%
% \endgroup
% }%

\input{lddef}

\makeatletter
\@ifclasswith{exam}{answers}{\newcommand{\firstblock}{comments_ldps1}}{\newcommand{\firstblock}{policies}}
\makeatother

\begin{document}

\pagestyle{headandfoot}
\runningheadrule


\newcounter{psctr}
\setcounter{psctr}{3} % set to the times of problem

\runningheader{Programming Assignment \thepsctr}
              {\textsc{Learning from Data}}
              { Page \thepage\ of \numpages}
\firstpagefooter{}{}{}
\runningfooter{}{}{}


\newcounter{Sequ}
\newenvironment{Sequation}
   {\stepcounter{Sequ}%
     \addtocounter{equation}{-1}%
     \renewcommand\theequation{S\arabic{Sequ}}\equation}
   {\endequation}
%\topskip0pt

% \vspace*{\fill}
\centering

% \vspace{0.3em}
\centering
\renewcommand{\thequestion}{\arabic{psctr}.\arabic{question}}
\courseheader

\begin{center}
  \underline{\bf Programming Assignment \thepsctr} \\
\end{center}
\begin{flushleft}
  \textbf{Issued:} \displaydate{issuedate} \hfill
  \textbf{Due:} \displaydate{duedate} 
\end{flushleft}

\hrule 

\input{\firstblock}

%\pointname{}
%\vspace{\footskip}
\vspace{1em}


%\pointname{}
%\vspace{\footskip}
%\vspace{1em}

\begin{questions}
\question (BP Neural Network) Suppose we are given a dataset $\{x^{(i)},y^{(i)}: i=1,2,...,m \}$ generated by
\begin{equation}
y^{(i)} = \sin x^{(i)} \quad \forall i
\end{equation}
Please design a neural network to represent this function using back-propagation. The structure of the network is:

\begin{enumerate}
\item Input layer, shape $N\times 1$, $N$ is batch size
\item Linear layer, shape $1\times 80$
\item ReLU activation layer
\item Linear layer, shape $80\times 1$

\end{enumerate}
\end{questions}
\begin{flushleft}
\textbf{Notice}: \\
\begin{enumerate}
\item Submit your codes and the results in image or pdf.
\item \textbf{DO NOT} change other part of codes apart from the given spaces for you to fill, the parameters are carefully tuned in advance.
\item The final result would be like Fig.\ref{fig1} if your code is right.

\begin{figure}[htbp]
\centering
\includegraphics[width=5in]{1.png}
\caption{Output result.}\label{fig1}
\end{figure}

\end{enumerate}
\end{flushleft}


\bibliographystyle{plain}
\bibliography{ref}

\end{document}
%%% Local Variables:
%%% mode: latex
%%% TeX-master: t
%%% End:

%  LocalWords:  headandfoot covariance
